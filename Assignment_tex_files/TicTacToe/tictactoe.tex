\documentclass{article}
\usepackage{fullpage}
\usepackage[utf8]{inputenc}
\title{Tic-Tac-Toe Instructions}
\author{Computersystemen}
\date{March 2017}

\usepackage{natbib}
\usepackage{graphicx}

\begin{document}

\maketitle

\section{Task 0: Tic-Tac-Toe}
In this assignment, you are to create a game-interface similar to Bratko's implementation for the game tic-tac-toe in Prolog. You are allowed to use 1 day to complete this assignment. To help you with this task, you can find two files from \citet{bratko2001prolog} on the blackboard that you can use as an example, namely:
\begin{enumerate}
    \item \textbf{minimax.pl}: minimax implementation.
    \item \textbf{alphabeta.pl}: $\alpha$-$\beta$ implementation.
\end{enumerate}

To complete these implementations to our Tic-Tac-Toe example, 4 relations have to be implemented, namely:

\begin{verbatim}
moves( Pos, PosList ) % Legal moves in Pos, fails when Pos is terminal
staticval( Pos, Val ) % value of a Terminal node (utility function)
min_to_move( Pos )    % the opponents turn
max_to_move( Pos )    % our turn
\end{verbatim}

You may program additional functions into your Prolog program, yet the 4 functions above should be working as intended.

Some things to think about before directly starting to code:
\begin{itemize}
    \item Perhaps read the relevant pages of \citet{bratko2001prolog}.
    \item How do you want to represent the state of a board?
    \item Think of the different win/lose conditions. (Rules vs hardcoding)
\end{itemize}

\section{Requirements}
Finally, a (simple) interface should be created to play the game against the computer. This interface has to:
\begin{itemize}
    \item Apply both minimax and $\alpha$-$\beta$ correctly for the computer.
    \item Allow the user to change who starts the game (user/computer).
    \item Display the board in a comprehensible way or at least provided good instructions.
    \item Allow a user to select his move, and only allow legal moves (e.g. you can not mark a spot that is already taken)
    \item Display to the user when it has ended in a win/loss/draw, end also actually end the game.
\end{itemize}

\section{Hand-in}
Hand in either a single prolog file on blackboard named as: \textless groupnumber\textgreater\_\textless name1\textgreater\_\textless name2\textgreater\_TTT.pl, or hand in all files in a zip/tar with the name: \textless groupnumber\textgreater\_\textless name1\textgreater\_\textless name2\textgreater\_TTT.zip
\bibliographystyle{plainnat}
\bibliography{references}
\end{document}
